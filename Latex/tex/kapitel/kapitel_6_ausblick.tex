\chapter{Zusammenfassung} \label{Kap6}

Das Ziel der Arbeit war es herauszuarbeiten, welche Kartierungsalgorithmen existieren und wie sich diese für den Pioneer 3-DX Roboter in das \ac{ROS} integrieren, konfigurieren, starten und verbessern lassen.

In \autoref{Kap3} wurden die Funktionsweise sowie die Verwendung im \ac{ROS} der einzelnen Kartierungsalgorithmen vorgestellt.

In \autoref{Kap4} wurden das eigene \ac{ROS}-Package vorgestellt, welches die Launch-Files und Konfigurationen der drei Kartierungsalgorithmen speichert. Außerdem wurde hier dargestellt, wie die Packages installiert werden und wie die Kartierung sowohl online als auch offline gestartet werden kann.

In \autoref{Kap5} wurden dann Tests nach zwei Testszenarien für die Kartierungsalgorithmen Cartographer, gmapping und hector\_mapping durchgeführt sowie ausgewertet. Dabei haben sich gewisse Unterschiede zwischen den Algorithmen herausarbeiten lassen. Es hat sich herausgestellt, dass der Cartographer für die ausgewählten Szenarien auf Grund der Konfigurationsmöglichkeiten und der guten Kartierung am besten abgeschnitten hat.