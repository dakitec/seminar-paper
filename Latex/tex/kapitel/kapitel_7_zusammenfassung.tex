\chapter{Ausblick} \label{Kap7}

Weiterführend hätte man noch das Thema der 2\nicefrac{1}{2}D-Kartierung aufgreifen können, bei der Elemente wie Treppenabgänge sowie Geländer berücksichtigt werden. Im nächsten Schritt wäre auch eine 3D-Kartierung von Bedeutung, welche es ermöglicht, mehr als ein Stockwerk zu kartieren. Interessant wäre es hierbei mit einem Höhensensor zu messen, in welchem Stockwerk der Roboter sich befindet oder allgemein nicht nur eine 2D-Karte aufzunehmen, sondern durch einen bewegbaren Laserscanner, welcher auch Höhendaten liefern kann, eine Karte zu erstellen. Dafür würde sich beispielsweise der Cartographer gut eignen, da dieser auch einige 3D-Optionen anbietet und grundsätzlich sehr viele Konfigurationsmöglichkeiten hat.

Des weiteren wäre es hilfreich gewesen, die \ac{IMU} vollständig zu kalibrieren und konfigurieren, um zu testen, ob diese den Cartographer-Algorithmus verbessert hätten.

In einem weiterführenden Schritt hätte man noch die Navigation nach dem Erstellen der Karten ausprobieren können. Dabei hätte man vergleichend auf verschiedene Navigationsalgorithmen eingehen können und diese ebenfalls wie die Kartierungsalgorithmen in einem \ac{ROS}-Package bündeln können.