% -------------------------------------------------------
% Daten für die Arbeit
% Wenn hier alles korrekt eingetragen wurde, wird das Titelblatt
% automatisch generiert. D.h. die Datei titelblatt.tex muss nicht mehr
% angepasst werden.

% Titel der Arbeit auf Deutsch
\newcommand{\hsmatitelde}{Untersuchung von Kartierungsalgorithmen unter ROS mit dem Pioneer 3-DX}

% Titel der Arbeit auf Englisch
\newcommand{\hsmatitelen}{Analysis of cartographing algorithms in ROS using the Pioneer 3-DX}

% Weitere Informationen zur Arbeit
\newcommand{\hsmaort}{Mannheim}    % Ort
\newcommand{\hsmaautorvname}{Daniel} % Vorname(n)
\newcommand{\hsmaautornname}{Koch} % Nachname(n)
\newcommand{\hsmadatum}{23.10.2019} % Datum der Abgabe
\newcommand{\hsmajahr}{2019} % Jahr der Abgabe
\newcommand{\hsmafirma}{} % Firma bei der die Arbeit durchgeführt wurde
\newcommand{\hsmabetreuer}{Prof. Dr. Thomas Ihme, Hochschule Mannheim} % Betreuer an der Hochschule
\newcommand{\hsmazweitkorrektor}{} % Betreuer im Unternehmen oder Zweitkorrektor
\newcommand{\hsmafakultaet}{I} % I für Informatik oder E, S, B, D, M, N, W, V
\newcommand{\hsmastudiengang}{IB} % IB IMB UIB IM MTB (weitere siehe titleblatt.tex)

% Zustimmung zur Veröffentlichung
\setboolean{hsmapublizieren}{true}   % Einer Veröffentlichung wird zugestimmt
\setboolean{hsmasperrvermerk}{false} % Die Arbeit hat keinen Sperrvermerk

% -------------------------------------------------------
% Abstract

% Kurze (maximal halbseitige) Beschreibung, worum es in der Arbeit geht auf Deutsch
\newcommand{\hsmaabstractde}{Diese Studienarbeit untersucht Kartierungsalgorithmen wie Google Cartographer, gmapping oder hector\_mapping im Robot Operating System (ROS), einem open-source Meta-Betriebssystem für die Entwicklung von Robotern. Es wird die Funktionsweise der genannten Algorithmen analysiert und ein Package aufgesetzt, welches mit dem Pioneer 3-DX kompatibel ist. Dieses Package wird genutzt, um die Fähigkeit der Algorithmen unter bestimmten Bedingungen in einer 2D-Umgebung zu testen. In diesen Tests schneidet der Google Cartographer auf Grund von genauen Kartierungen und einer Vielzahl an Konfigurationsmöglichkeiten am besten ab.}

% Kurze (maximal halbseitige) Beschreibung, worum es in der Arbeit geht auf Englisch
\newcommand{\hsmaabstracten}{This study evaluates cartographing algorithms like Google Cartographer, gmapping or hector\_mapping within the Robot Operating System (ROS). ROS is an open-source meta operating system for the development of robots. The study analyzes the functionality of these algorithms and sets up a package which is compatible with the Puoneer 3-DX. The package is used then to evaluate the capability of the algorithms in a defined 2D environment. In these tests the Google Cartographer performs best since it creates the most accurate maps and had the most configuration options.}
